\documentclass[1pt]{beamer}
\usetheme{Warsaw}
\usepackage[utf8]{inputenc}
\usepackage[russian]{babel}
\usepackage[OT1]{fontenc}
\usepackage{amsmath}
\usepackage{amsfonts}
\usepackage{amssymb}
\usepackage{graphicx}

\usepackage{appendixnumberbeamer}
 
\newtheorem{rustheorem}{Теорема }
\DeclareMathOperator*{\argmin}{argmin}
\let\T\intercal
\let\ov\overline
\def\bar_#1{\bold{\ov{#1}}}
\def\hatup_#1{\bold{\hat{#1}}}
\def\ton_#1{1,2,\dots,#1}
\def\msf_#1{\mathsf{#1}}
\def\Lagr{\mathcal{L}}
\def\bo_#1{\bold{#1}}

\setbeamersize{text margin left=10pt,text margin right=10pt} 

\setbeamertemplate{footline}
{
  \leavevmode%
  \hbox{%
  \begin{beamercolorbox}[wd=.25\paperwidth,ht=2.25ex,dp=1ex,center]{author in head/foot}%
    \usebeamerfont{author in head/foot}\insertshortauthor
  \end{beamercolorbox}%
  \begin{beamercolorbox}[wd=.75\paperwidth,ht=2.25ex,dp=1ex,center]{title in head/foot}%
    \usebeamerfont{title in head/foot}\insertshorttitle\hspace*{3em}
  \end{beamercolorbox}}%
  \vskip0pt%
}


\makeatletter
\setbeamertemplate{headline}
{%
  \leavevmode%
  \@tempdimb=2.4375ex%
  \ifnum\beamer@subsectionmax<\beamer@sectionmax%
    \multiply\@tempdimb by\beamer@sectionmax%
  \else%
    \multiply\@tempdimb by\beamer@subsectionmax%
  \fi%
  \ifdim\@tempdimb>0pt%
    \advance\@tempdimb by 1.825ex%
    \begin{beamercolorbox}[wd=.25\paperwidth,ht=\@tempdimb]{section in head/foot}%
      \vbox to\@tempdimb{\vfil\insertsectionnavigation{.25\paperwidth}\vfil}%
    \end{beamercolorbox}%
    \begin{beamercolorbox}[wd=.75\paperwidth,ht=\@tempdimb]{subsection in head/foot}%
      \vbox to\@tempdimb{\vfil\insertsubsectionnavigation{.75\paperwidth}\vfil}%
    \end{beamercolorbox}%
  \fi%
}

\author{Андрей Квасов}
\title{Анализ динамики изменения инвестиционного портфеля}
%\setbeamercovered{transparent} 
\setbeamertemplate{navigation symbols}{} 
%\logo{} 
\institute{МГУ им. М.\ В.\ Ломоносова} 
\date{24 декабря 2015} 
%\subject{} 
\begin{document}

\begin{frame}
\titlepage
\end{frame}

%\begin{frame}
%\tableofcontents
%\end{frame}


\begin{frame}{Обзор предметной области и постановка задачи}
\begin{itemize}
\item Объектом распознавания являются сигналы $\bo_y = (\bo_y_1,\dots,\bo_y_N), \bo_y_i \in \mathbb{R}^n, i=1,\dots,N$. Целевая переменная - $\bo_y^0 \in \mathbb{R}^N$.
\item Рассматривается модель линейной регрессии, где искомой величиной является сигнал $\bo_x = (\bo_x_1,\dots,\bo_x_N)$, связанный с предыдущими  $\bo_y^0_t = \bo_y_t^\T\bo_x_t + \varepsilon_t$. Задача состоит в нахождении по заданному сигналу объекта $\bo_y$ и целевого вектора $\bo_y^0$ объекта, его регрессионных коэффициентов $\bo_x$.
\end{itemize}
\end{frame}

\begin{frame}{Инвестиционный портфель хэдж фонда}
\framesubtitle{Постановка задачи}
\begin{columns}[T]
    \begin{column}{.5\textwidth}
     \begin{block}{Returns Based Style Analysis model}
		$z_t^i$ - цена актива в портфеле\\
		$m_t^i$ - количество активов в портфеле, данного вида\\
		$y_t^i = \frac{z_t^i - z_{t-1}^i}{z_{t-1}^i}$ - доходности активов в портфеле\\
    \end{block}
    \end{column}
    \begin{column}{.5\textwidth}
   	 \includegraphics[width=\textwidth]{pic/pic1.png}
    \end{column}
  \end{columns}
\end{frame}
	
\begin{frame}{Инвестиционный портфель хэдж фонда}
\framesubtitle{Постановка задачи}
\begin{columns}[T]
    \begin{column}{.5\textwidth}
     \begin{block}{Returns Based Style Analysis model}
		$\hat{y}_t^0 = \sum_{i=1}^n m_t^iz_t^i$ - стоимость инвестиционного портфеля (скрытая величина)\\
		$y_t^0 = \frac{\hat{y}_t^0 - \hat{y}_{t-1}^0}{\hat{y}_{t-1}^0}$ - доходность портфеля (наблюдаемая величина)\\
		$\hat{x}_t^i = \frac{m_t^i z_t^i}{\sum_{i=1}^n m_t^i z_t^i}$ - доли активов в инвестиционном портфеле. доходности активов в портфеле\\
    \end{block}
    \end{column}
    \begin{column}{.5\textwidth}
   	 \includegraphics[width=\textwidth]{pic/pic1.png}\\
   	\qquad\qquad \[
\begin{cases}
y_t^0 = \bo_y_t^\T \bo_x_t + \varepsilon_t^1, \text{ где } \bo_x_t \approx \bo_{\hat{x}}_t \\
\sum_{i=1}^n x_t^i = 1
\end{cases}
\]
    \end{column}
  \end{columns}
\end{frame}


\begin{frame}{Математическая модель инвестиционного портфеля хэдж фонда}
\begin{block}{Модель наблюдения: узловые функции $\psi_t(\bo_x_t)$}
\begin{align*}
&\psi_t(\bo_x_t) = (y_t^0 - \bo_y_t^\T \bo_x_t)^2 = (\bold{x}_t - \bold x_t^0)^\T \msf_Q_t^0(\bold x_t - \bold x_t^0), \text{где }\msf_Q_t^0 = \bo_y_t \bo_y_t^\T,\ \bo_x_t^0 = \frac{y_t^0}{\bo_y_t^\T \bo_y_t} \bo_y_t\\
\end{align*}
\end{block}

\begin{block}{Модель динамики: функции связи $\gamma_t(\bo_x_{t-1}, \bo_x_t)$}
\begin{align*}
&\gamma_t(\bo_x_{t-1}, \bo_x_t) = (\bold x_t - \msf_A_t \bold x_{t-1})^\T \msf_U_t (\bold x_t - \msf_A_t \bold x_{t-1})\\
&\msf_U = \msf_{Diagonal}(\alpha), \text{где }\alpha - \text{ параметр сглаживания}\\
&\bo_x_t = \frac{1+\bo_y_t}{1+\bo_x_{t-1}^\T\bo_y_t} \Rightarrow\msf_A_t = \msf_{Diagonal}(\frac{1+\bo_y_t}{1+y_t^0}),\ \bo_x_t = \msf_A_t\bo_x_{t-1} + \varepsilon_t^2
\end{align*}
\end{block}
\end{frame}

\begin{frame}{Математическая модель инвестиционного портфеля хэдж фонда}
\framesubtitle{Общий критерий для оценивания состава портфеля}
\begin{block}{Общий критерий для оценивания состава портфеля}
\begin{align*}
&\msf_J(\bold x_1, \dots, \bold x_N) =\sum_{t=1}^N \psi(\bold x_t) + \sum_{t=2}^N \gamma(\bold x_{t-1}, \bold x_t) \\
&=\sum_{t=1}^N (\bold{x}_t - \bold x_t^0)^\T \msf_Q_t^0(\bold x_t - \bold x_t^0) + \sum_{t=2}^N (\bold x_t - \msf_A_t \bold x_{t-1})^\T \msf_U_t (\bold x_t - \msf_A_t \bold x_{t-1})
\end{align*}
\end{block}
\begin{block}{Задача программирования с парно-сепарабельной функцией}
\[\begin{cases}
\msf_J(\bold x_1, \dots, \bold x_N)\rightarrow\min_{\bo_x_1,\dots,\bo_x_N}\\
\sum_{i=1}^n x_t^i = 1,\ t=1,\dots,N
\end{cases}
\]
\end{block}
\end{frame}


\begin{frame}{Математическая модель инвестиционного портфеля хэдж фонда}
\framesubtitle{Учет ограничения на сумму долей активов} %[r]
\begin{align*}
&\text{Запишем ограничения равенства векторов }\bold x_t = (x_t^1,\dots,x_t^n) \in \mathbb{R}^n:\\
&1 - \sum_{i=1}^{n}x_t^i = 0\ \Rightarrow\ x_t^n = 1 - \sum_{i=1}^{n-1}x_t^i\ \Rightarrow\ \bar_x_t = (x_t^1,\dots,x_t^{n-1})\in \mathbb{R}^{n-1}\\
&\text{Где } \bold x_t = \msf_P\bar_x_t + \msf_p, \\
&\msf_P = \begin{pmatrix}
	1 & 0 & 0 &\dots & 0\\
	0 & 1 & 0 & \dots & 0 \\
	\vdots & \vdots & \ddots & \dots & \vdots\\
	0 & 0 & 0 & \ddots & 0 \\
	0 & 0 & 0 & \dots & 1 \\
	-1 &-1 &-1 & \dots &-1 \\
	\end{pmatrix} \in \mathbb{R}^{n \times (n-1)}
\qquad \msf_p = \begin{pmatrix} 0 \\ \vdots \\ 0 \\ 1	\end{pmatrix} \in \mathbb{R}^n
\end{align*}
\end{frame}


\begin{frame}{Математическая модель инвестиционного портфеля хэдж фонда}
\framesubtitle{Учет ограничения на сумму долей активов}
\begin{align*}
&\text{Преобразуем другие переменные :}\\
&\bar_y_t = \msf_P^\T \bold y_t,\ \bar_y_t \in \mathbb{R}^{n-1}\\
&\overline{y}_t^0 = y_t^0 - \msf_p^\T\bold y_t = y_t^0 - y_t^n,\ \overline{y}_t^0 \in \mathbb{R}\\
&\text{Тогда модель нестационарной регрессии преобразуется в вид:}\\
&\overline{y}_t^0 = \bar_x_t^\T\bar_y_t + \xi_t,\ \bar_x_t, \bar_y_t \in \mathbb{R}^{n-1};\ \text{Также } \overline{\msf_Q}_t^0 = \bar_y_t \bar_y_t^\T,\ \bar_x_t^0 = \frac{\overline{y}_t^0}{\bar_y_t^\T \bar_y_t} \bar_y_t\\
&\msf_J(\bold x_1, \dots, \bold x_N) =\sum_{t=1}^N (\bar_x_t - \bar_x_t^0)^\T \overline{\msf_Q}_t^0(\bar_x_t - \bar_x_t^0) + \\
&+ \sum_{t=2}^N \left[\msf_A_t (\msf_P\bar_x_{t-1} + \msf_p) - (\msf_P\bar_x_t + \msf_p)\right]^\T \msf_U_t \left[\msf_A_t (\msf_P\bar_x_{t-1} + \msf_p) - (\msf_P\bar_x_t + \msf_p)\right]
\end{align*}
\end{frame}

\begin{frame}{Математическая модель инвестиционного портфеля хэдж фонда}
\framesubtitle{Учет ограничения на сумму долей активов}
\begin{align*}
&\text{Рассмотрим функцию связи на момент $t$:}\\
&\left(\msf_A_t (\msf_P\bar_x_{t-1} + \msf_p) - (\msf_P\bar_x_t + \msf_p)\right)^\T \msf_U_t \left(\msf_A_t (\msf_P\bar_x_{t-1} + \msf_p) - (\msf_P\bar_x_t + \msf_p)\right) =\\ &=\bar_x_{t-1}^\T\msf_P^\T\msf_A_t^\T\msf_U_t\msf_A_t\msf_P
\bar_x_{t-1} + 2\msf_p^\T\msf_A_t^\T\msf_U_t\msf_A_t\msf_P\bar_x_{t-1}+const(\msf_p)+\\
&+\bar_x_t^\T\msf_P^\T\msf_U_t\msf_P\bar_x_t+2\msf_p^\T\msf_U_t\msf_P\bar_x_t+const(p)-
2\bar_x_{t-1}^\T\msf_P^\T\msf_A_t^\T\msf_U_t\msf_P\bar_x_t-\\
&-2\msf_p^\T\msf_A_t^\T\msf_U_t\msf_P\bar_x_t-
2\msf_p^\T\msf_U_t\msf_A_t\msf_P\bar_x_{t-1}-const(p)\\
& \text{Найдем минимум новой функции связи:}\\
&\gamma(\bar_x_{t-1}, \bar_x_t) = \bar_x_{t-1}^\T\msf_P^\T\msf_A_{t}^\T\msf_U_{t}\msf_A_{t}\msf_P
\bar_x_{t-1} +\bar_x_t^\T\msf_P^\T\msf_U_t\msf_P\bar_x_t-2\bar_x_{t-1}^\T\msf_P^\T\msf_A_t^\T\msf_U_t\msf_P\bar_x_t\\
&\text{Условие стационарной точки: } \msf_P^\T\msf_U_t\msf_P\bar_x_t-\msf_P^\T\msf_U_t\msf_A_t\msf_P\bar_x_{t-1} = 0\Rightarrow\\
&\Rightarrow \bar_x_t = (\msf_P^\T\msf_U_t\msf_P)^{-1}\msf_P^\T\msf_U_t\msf_A_t\msf_P\bar_x_{t-1} \\
&\text{Обозначим }\overline{\msf_A}_t = (\msf_P^\T\msf_U_t\msf_P)^{-1}\msf_P^\T\msf_U_t\msf_A_t\msf_P;\ \overline{\msf_U}_t = \msf_P^\T\msf_U_t\msf_P\\
\end{align*}
\end{frame}


\begin{frame}{Математическая модель инвестиционного портфеля хэдж фонда}
\framesubtitle{Учет ограничения на сумму долей активов}
\begin{align*}
&\text{Рассмотрим новую узловую функцию на момент t,}\\ &\text{с учетом оставшихся слагаемых: }
\hat\psi(\bar_x_t) = (\bar_x_t - \bar_x_t^0)^\T \bar_{\msf_Q}_t^0 (\bar_x_t - \bar_x_t^0) + \\
&+2\msf_p^\T\msf_A_{t+1}^\T\msf_U_{t+1}\msf_A_{t+1}\msf_P\bar_x_t +2\msf_p^\T\msf_U_t\msf_P\bar_x_t-2\msf_p^\T\msf_A_t^\T\msf_U_t\msf_P\bar_x_t-
2\msf_p^\T\msf_U_{t+1}\msf_A_{t+1}\msf_P\bar_x_t\\
&\text{Приведем новую узловую функцию к виду: }\\
&\hat\psi(\bar_x_t)=(\bar_x_t - \hatup_x_t^0)^\T\hatup_{\msf_Q}_t^0(\bar_x_t - \hatup_x_t^0)=(\bar_x_t - \bar_x_t^0)^\T \bar_{\msf_Q}_t^0 (\bar_x_t - \bar_x_t^0) - 2\msf_p^\T\msf_S\msf_P\bar_x_t,\text{где}\\
&\msf_S = \msf_A_t^\T\msf_U_t+
\msf_U_{t+1}\msf_A_{t+1}-(\msf_A_{t+1}^\T\msf_U_{t+1}\msf_A_{t+1}+\msf_U_t)\\
&\text{Раскрывая скобки получим: }
{\bar_x_t}^\T\hatup_{\msf_Q}_t^0\bar_x_t = {\bar_x_t}^\T\bar_{\msf_Q}_t^0\bar_x_t\Rightarrow\hatup_{\msf_Q}_t^0=\bar_{\msf_Q}_t^0
=\bar_y_t\bar_y_t^\T\\
&\hatup_x_t^0{}^\T\bar_{\msf_Q}_t^0\bar_x_t =\bar_x_t^0{}^\T\bar_{\msf_Q}_t^0\bar_x_t +  \msf_p^\T\msf_S\msf_P\bar_x_t\Rightarrow
(\hatup_x_t^0-\bar_x_t^0)^\T\bar_y_t\bar_y_t^\T = \msf_p^\T\msf_S\msf_P |\cdot\bar_y_t\Rightarrow\\
&(\hatup_x_t^0-\bar_x_t^0)^\T\bar_y_t=(\bar_y_t^\T\bar_y_t)^{-1}\msf_p^\T\msf_S\msf_P\bar_y_t.\text{ Таким образом }\hatup_x_t^0 =\bar_x_t^0 + \cfrac1{\bar_y_t^\T\bar_y_t}\msf_P^\T\msf_S^\T\msf_p
\end{align*}
\end{frame}

\begin{frame}{Математическая модель инвестиционного портфеля хэдж фонда}
\framesubtitle{Учет ограничения на сумму долей активов}
\begin{block}{Общий критерий для оценивания состава портфеля}
\begin{align*}
\msf_J(\bar_x_1, \dots, \bar_x_N) =&\sum_{t=2}^N (\bar_x_t - \bar_x_t^0)^\T \overline{\msf_Q}_t^0(\bar_x_t - \bar_x_t^0) + \\&\sum_{t=2}^N (\overline{\msf_A}_t \bar_x_{t-1} - \bar_x_t)^\T \overline{\msf_U}_t (\overline{\msf_A}_t \bar_x_{t-1} - \bar_x_t)\\
&\overline{\msf_Q}_t^0 = \bar_y_t \bar_y_t^\T,\ \bar_x_t^0 = \frac{\overline{y}_t^0}{\bar_y_t^\T \bar_y_t} \bar_y_t + \msf_P^\T(\msf_U_t\msf_A_t + \msf_U_{t+1}\msf_A_{t+1})\msf_p
\end{align*}
\end{block}
\begin{block}{Задача программирования с парно-сепарабельной функцией}
\[\begin{cases}
\msf_J(\bar_x_1, \dots, \bar_x_N)\rightarrow\min_{\bar_x_1,\dots,\bar_x_N}\\
\sum_{i=1}^n \overline{x}_t^i = 1,\ t=1,\dots,N
\end{cases}
\]
\end{block}
\end{frame}

\begin{frame}{Алгоритм оценивания состава инвестиционного портфеля портфеля}
\framesubtitle{Алгоритм динамического программирования}
\begin{columns}[T]
\begin{column}{.5\textwidth}
   	  \begin{block}{Левые функции Беллмана}
   	  \begin{itemize}
   	  \item Частичный критерий:\ 
		$\msf_J_t^{-}(\bar_x_1, \dots, \bar_x_t) = \sum_{i=1}^t \psi_i(\bar_x_i) + \sum_{i=2}^t \gamma_i(\bar_x_{i-1}, \bar_x_i)$
		\item Функция Беллмана:\\
		$\tilde{\msf_J}_t^{-}(\bar_x_t) = \displaystyle\min_{\bar_x_1,\dots,\bar_x_{t-1}}\msf_J_t^{-}(\bar_x_1, \dots, \bar_x_t) $
		\item Рекуррентное соотношение:\
		$\tilde{\msf_J}_t^{-}(\bar_x_t) =\psi_t(\bar_x_t) +  \displaystyle\min_{\bar_x_{t-1}}[\gamma_t(\bar_x_{t-1}, \bar_x_t)+\tilde{\msf_J}_{t-1}^{-}(\bar_x_{t-1})]$
		 \end{itemize}
    \end{block}
    \end{column}
    \begin{column}{.5\textwidth}
     \begin{block}{Правые функции Беллмана}
     \begin{itemize}
   	  \item Частичный критерий:\ 
		$\msf_J_t^{+}(\bar_x_t, \dots, \bar_x_N) = \sum_{i=t}^N \psi_i(\bar_x_i) + \sum_{i=t+1}^N \gamma_i(\bar_x_{i-1}, \bar_x_i)$
		\item Функция Беллмана:\
		$\tilde{\msf_J}_t^{+}(\bar_x_t) = \displaystyle\min_{\bar_x_{t+1},\dots,\bar_x_N}\msf_J_t^{+}(\bar_x_t, \dots, \bar_x_N)$
		\item Рекуррентное соотношение:\
		$\tilde{\msf_J}_t^{+}(\bar_x_t) =\psi_t(\bar_x_t) +  \displaystyle\min_{\bar_x_{t+1}}[\gamma_{t+1}(\bar_x_{t}, \bar_x_{t+1})+\tilde{\msf_J}_{t+1}^{-}(\bar_x_{t+1})]$
		\end{itemize}
    \end{block}
    \end{column}
\end{columns}
\qquad Искомые значения $\bar_x_t = \displaystyle\min_{\bar_x_t}[\tilde{\msf_J}_t^{-}(\bar_x_t) + \tilde{\msf_J}_t^{+}(\bar_x_t) - \psi(\bar_x_t)]$
\end{frame}


\begin{frame}{Алгоритм оценивания состава портфеля}
\framesubtitle{Пересчет параметров квадратичных форм}
\begin{rustheorem}[Канонический вид квадратичной функции]
	Пусть $\msf_L(\bo_x) = \sum_{i=1}^M(\msf_D_i\bo_x - \bo_z_i)^\T\msf_B_i(\msf_D_i\bo_x - \bo_z_i), \bo_x\in\mathbb{R}^n,$ такие что $\bo_z_i\in\mathbb{R}^{k_i},\ k_i\leq n,$\\
		$\msf_B_i\leq 0\  (\msf_B_i\in\mathbb{R}^{n\times n},\msf_D_i\in\mathbb{R}^{k_i\times n})$, так что $\exists i: \msf_B_i>0, rg(\msf_D_i)=k_i.$\\
\begin{itemize}
\item Тогда можно представить $\msf_L(\bo_x) = (\bo_x  - \bo_z)^\T\msf_B(\bo_x-\bo_z)+\msf_d,$\\
$\msf_B = \sum_{i=1}^M \msf_D_i^\T\msf_B_i\msf_D_i>0;\ z=\displaystyle\argmin_{\bo_x\in\mathbb{R}^n}L(x)={\msf_B}^{-1}\sum_{i=1}^M
		\msf_D_i^\T\msf_B_i\bo_z_i;$\\
$d = \displaystyle\min_{\bo_x\in\mathbb{R}^n}=\sum_{i=1}^M(\bo_z_i-\msf_D_i\bo_z)^\T\msf_B_i\bo_z_i\geq0;$\\
\item $\text{Если }M=2, k_1=k_2=n, D_1, D_2$ --- невырожденные, то
\begin{align*}
&\msf_d=(\msf_D_1^{-1}\bo_z_1-\msf_D_2^{-1}\bo_z_2)^\T(\msf_D_1^{-1}\msf_B_1^{-1}(\msf_D_1^{-1})^\T+\msf_D_2^{-1}\msf_B_2^{-1}(\msf_D_2^{-1})^\T)^{-1}\cdot\\
		&\cdot(\msf_D_1^{-1}\bo_z_1-\msf_D_2^{-1}\bo_z_2)
\end{align*}
\end{itemize}
\end{rustheorem}
\end{frame}


\begin{frame}{Алгоритм оценивания состава инвестиционного портфеля портфеля}
\framesubtitle{Алгоритм динамического программирования}
($\bo_x_t := \bar_x_t$)\\ Квадратичный вид функций Беллмана: \[\begin{cases}\tilde{\msf_J}_t^{-}(\bo_x_t) = (\bo_x_t - \tilde{\bo_x}_t^{-})^\T\tilde{\msf_Q}_t^{-}(\bo_x_t - \tilde{\bo_x}_t^{-}) + \tilde{b}^{-}\\ \tilde{\msf_J}_t^{+}(\bo_x_t) = (\bo_x_t - \tilde{\bo_x}_t^{+})^\T\tilde{\msf_Q}_t^{+}(\bo_x_t - \tilde{\bo_x}_t^{+}) + \tilde{b}^{+}\end{cases}\]
\begin{block}{Пересчет параметров левой функции Беллмана}
\begin{align*}
&\tilde{b}_1^{-}=0,\ \tilde{\bo_x}_1^{-} = \bo_x_1^0, \tilde{\msf_Q}_1^{-} = \msf_Q_1^0\\
&\tilde{\msf_Q}_t^{-} =\msf_Q_t^0 + (\msf_U^{-1}+\msf_A_t^\T\tilde{\msf_Q}_{t-1}^{-}{}^{-1}\msf_A_t)^{-1} \\
&\tilde{\bo_x}_t^{-} = \tilde{\msf_Q}_t^{-}{}^{-1}(\msf_Q_t^0\bo_x_t^0 + (\msf_U^{-1}+\msf_A_t^\T\tilde{\msf_Q}_{t-1}^{-}{}^{-1}\msf_A_t)^{-1}\msf_A_t\tilde{\bo_x}_{t-1}^{-})\\
&\tilde{b}_t^{-} = \tilde{b}_{t-1}^{-} + (\bo_x_t^0 - \tilde{\bo_x}_t^{-})^\T\msf_Q_t^0\bo_x_t^0 + (\tilde{\bo_x}_{t-1}^{-} - \msf_A_t^\T\tilde{\bo_x}_t^{-})(\msf_U^{-1}+\msf_A_t^\T\tilde{\msf_Q}_{t-1}^{-}{}^{-1}\msf_A_t)^{-1})\tilde{\bo_x}_{t-1}^{-} \\
\end{align*}
\end{block}
\end{frame}


\begin{frame}{Алгоритм оценивания состава инвестиционного портфеля портфеля}
\framesubtitle{Алгоритм динамического программирования}
\begin{block}{Пересчет параметров правой функции Беллмана}
\begin{align*}
&\tilde{b}_N^{+}=0,\ \tilde{\bo_x}_N^{+} = \bo_x_N^0, \tilde{\msf_Q}_N^{+} = \msf_Q_N^0\\
&\tilde{\msf_Q}_t^{+} =\msf_Q_t^0 + \msf_U\msf_A_{t+1}(\msf_U+\tilde{\msf_Q}_{t+1}^{+})^{-1}\tilde{\msf_Q}_{t+1}^{+}\msf_A_{t+1}^{-1}\\
&\tilde{\bo_x}_t^{+} = \tilde{\msf_Q}_t^{+}{}^{-1}(\msf_Q_t^0\bo_x_t^0 + \msf_U\msf_A_{t+1}(\msf_U+\tilde{\msf_Q}_{t+1}^{+})^{-1}\tilde{\msf_Q}_{t+1}^{+}\tilde{\bo_x}_{t+1}^{+})\\
&\tilde{b}_t^{+} = \tilde{b}_{t-1}^{+} + (\bo_x_t^0 - \tilde{\bo_x}_t^{-})^\T\msf_Q_t^0\bo_x_t^0 + (\tilde{\bo_x}_{t+1}^{-} - \msf_A_{t+1}\tilde{\bo_x}_t^{+})^\T\msf_U(\msf_U+
\tilde{\msf_Q}_{t+1}^{+})^{-1}\tilde{\msf_Q}_{t+1}^{+}\tilde{\bo_x}_{t+1}^{+}\\
\end{align*}
\end{block}
\end{frame}

\begin{frame}
\end{frame}

%\begin{frame}
%\end{frame}

\end{document}